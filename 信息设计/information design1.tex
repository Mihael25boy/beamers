\documentclass{beamer}
\usepackage{ctex, hyperref}
\usepackage[T1]{fontenc}

% other packages
\usepackage{latexsym,amsmath,xcolor,multicol,booktabs,calligra}
\usepackage{graphicx,pstricks,listings,stackengine,newtxtext,newtxmath}

\author{赵万春}
\title{信息设计导论}
\subtitle{贝叶斯劝说框架与进展}
\institute{天津财经大学 金融学院}
\date{2024年5月24日}
\usepackage{tufe}

% defs
\def\cmd#1{\texttt{\color{red}\footnotesize $\backslash$#1}}
\def\env#1{\texttt{\color{blue}\footnotesize #1}}
\definecolor{deepblue}{rgb}{0,0,0.5}
\definecolor{deepred}{rgb}{0.6,0,0}
\definecolor{deepgreen}{rgb}{0,0.5,0}
\definecolor{halfgray}{gray}{0.55}

\lstset{
	basicstyle=\ttfamily\small,
	keywordstyle=\bfseries\color{deepblue},
	emphstyle=\ttfamily\color{deepred},    % Custom highlighting style
	stringstyle=\color{deepgreen},
	numbers=left,
	numberstyle=\small\color{halfgray},
	rulesepcolor=\color{red!20!green!20!blue!20},
	frame=shadowbox,
}
\usefonttheme[onlymath]{serif}

\begin{document}

\kaishu
\section{基础文献}
\begin{frame}{第一部分:入门}
	
\end{frame}

\section{例子:KG2011}

\subsection{问题引入与设定}

\begin{frame}{问题}
	法官(judge)、检察官(prosecutor)和嫌疑人(defendant)
	\begin{itemize}
		\item 法官的目标是公平断案
		\item 检察官的目标是将犯人判为有罪
	\end{itemize}
	检察官可以调查嫌疑人,获取私人信息。\\
	\begin{itemize}
		\item 如果有罪,检察官如实汇报对自己有利
		\item 如果无罪,检察官如实汇报会损害自己效用
	\end{itemize}
	那么,检察官能否通过设计信息结构,提高理性法官的平均定罪概率?
\end{frame}

\begin{frame}{基础设定:行动集、支付与先验概率}
	嫌疑人有两种状态:有罪(Guilty)、无罪(Innocent)\\
	$$\Omega=\left\lbrace G,I\right\rbrace $$
	法官(Receiver):定罪(convict)或释放(acquit)\\
	法官的效用函数:
	\begin{itemize}
		\item[] 审判结果与状态相符时为1
		\item[] 审判结果与状态不符时为0
	\end{itemize}
	检察官(Sender)的效用函数取决于法官的行为:
	\begin{itemize}
		\item[] 法官选择“定罪”时为1
		\item[] 法官选择“释放”时为0
	\end{itemize}
	检察官的效用与事实情况无关,只和法官行为有关。\\
	检察官与法官共享先验信念Pr$\left(G\right)=0.3$、Pr$\left(I\right)=0.7$

\end{frame}

\begin{frame}{检察官行为、时点设定与基准模型}
	检察官可以调查,即选择一个信息结构$\pi: \Omega\to\Delta\left(S\right)$。\\~\\
	检察官调查$\to$检察官向法官告知结果$\to$法官根据结果做出决策\\~\\
	基准模型1:完全如实汇报,即令$S=G$,\\
	如果嫌疑人有罪(guilty,$ G $),检察官必然告知法官有罪,即$\pi\left(G\mid G\right)=1$;\\
	如果嫌疑人无罪(innocent,$ I $),检察官必然告知法官无罪,即$\pi\left(I\mid I\right)=1$。\\
	此时法官的支付恒为1\\
	检察官的收益是嫌疑人有罪的先验概率0.3。
\end{frame}

\begin{frame}{基准模型2:不进行任何调查}
	无论是否有罪,检察官都告知法官同一个证据$S=\left\lbrace 1\right\rbrace $。即$\pi\left(1\mid G\right)=\pi\left(1\mid I\right)=1$\\
	如果法官无论收到什么信号都判有罪,法官的期望收益为\\
	$\Pr(G)\times1+\Pr(I)\times0=0.3$\\
	如果法官无论收到什么信号都判无罪,法官的期望收益为\\
	$\Pr(G)\times0+\Pr(I)\times1=0.7$\\\pause
	因此,法官必然选择判无罪,检察官支付为0。\pause\\~\\
	检察官选择如实披露比完全不披露更有利。
\end{frame}

\subsection{贝叶斯劝说}

\begin{frame}{检察官能否做得更好?}
	令$S=\left\lbrace g,i\right\rbrace $,$\pi$满足
	$$\pi\left( g\mid G \right)=1, \pi\left( i\mid G \right)=0$$
	$$\pi\left( g\mid I \right)=\frac{3}{7}, \pi\left( i\mid I \right)=\frac{4}{7}$$\pause
	此时,法官收到信号$g$并判有罪时,所获支付为\pause

	$$judge's \ payoff=\frac{\overset{\text{有罪时收到信号g的概率}}{\pi\left( g \mid G\right) \Pr\left( G\right)} }{\underset{\text{有罪时收到信号g的概率}}{\pi\left( g \mid G\right) \Pr\left( G\right)}+\underset{\text{无罪时收到信号g的概率}}{\pi\left( g \mid I\right) \Pr\left( I\right)}}$$\pause
	$$=\frac{1\times0.3}{1\times 0.3+\frac{3}{7}\times 0.7}=\frac{1}{2}$$
\end{frame}

\begin{frame}{检察官能否做得更好?}
	法官收到信号$g$并判无罪时,所获支付为
	$$judge's \ payoff=\frac{\overset{\text{无罪时收到信号g的概率}}{\pi\left( g \mid I\right) \Pr\left( I\right)} }{\underset{\text{有罪时收到信号g的概率}}{\pi\left( g \mid G\right) \Pr\left( G\right)}+\underset{\text{无罪时收到信号g的概率}}{\pi\left( g \mid I\right) \Pr\left( I\right)}}$$\pause
	$$=\frac{\frac{3}{7}\times 0.7}{1\times 0.3+\frac{3}{7}\times 0.7}=\frac{1}{2}$$\pause
	如果收到信号$i$,法官可以确定嫌疑人无罪。\pause\\
	\alert{假定判有罪和无罪的收益相同时,法官会选择对检察官有利的选项}。
\end{frame}

\begin{frame}{检察官的收益}
	检察官的收益完全取决于发出信号$g$的概率,即
	$$prosecutor's payoff=\underset{\text{有罪时收到信号g的概率}}{\pi\left( g \mid G\right) \Pr\left( G\right)}+\underset{\text{无罪时收到信号g的概率}}{\pi\left( g \mid I\right) \Pr\left( I\right)}=0.6$$
	因此,只要检察官发出有罪信号,法官一定会判有罪。此时检察官的收益为0.6。\pause\\
	可以看出,不同情境下检察官的支付满足:\\
	什么都不披露$\left(0\right)<$完全披露$\left( 0.3\right) <$选择性披露$ \left( 0.6\right) $ 
\end{frame}

\section{正规表述}

\subsection{符号表达}

\begin{frame}{主体}
	信息发送者(Sender)和信息接受者(Receiver)\\
	状态空间$\Omega$是有穷的,且无论是S还是R都观测不到具体的状态。\\
	S和R都了解先验概率$p\in\Delta\left( \Omega \right) $\\
	接受者可以采取行动所构成的集合$A$是紧集。\\
	信息接受者R的效用函数:$A\ \times \ \Omega \ \to \mathbb{R}$满足连续性。\\
	信息发送者S的效用函数:$A\ \times \ \Omega \ \to \mathbb{R}$满足连续性。
\end{frame}

\begin{frame}{时间线与partial implementation}
	\begin{itemize}
		\item[1] 信息发送者S选择信息结构$\overset{\text{给定}\Omega\text{,以一定概率发送信号}S}{\pi:\Omega \to \Delta\left( S\right)} $
		\item[2] 信息接受者R观测到信号$S$的实现,并进行贝叶斯更新,采取行动$a\in A$
	\end{itemize}
	解概念:子博弈精炼均衡/子博弈完美均衡。\\~\\
	定义:信号发送者偏好的子博弈完美均衡:当信号接受者R的两个选项无差异时,接受者会选择更有利于信号发送者的选项。
\end{frame}

\begin{frame}{信念更新:后验信念}
	考虑任何信号发送者偏好的子博弈完美均衡(Sender-preferred subgame perfect equilibrium)\\
	给定后验信念$q\in\Delta\left( \Omega\right) $
	令集合$A\left( q\right) \subset A$为最大化信号接受者$R$期望收益的行动集合。(存在性源于Weierstrass theorem)
	$$A\left( q\right) =\underset{a\in A}{\arg\max}\ \  E_{q}\left[ u_{R} \left( a,\omega\right) \right] $$\pause 
	给定信念为$q$,最大化信息接受者期望收益的集合。对不同$\omega$各自可能性相乘取期望,采取行动$a$最大化收益。即
	$$E_{q}\left[ u_{R} \left( a,\omega\right) \right]=\sum_{\omega\in\Omega}\ \underset{\omega\text{发生的概率}}{ q\left(\omega\right)} \underset{\omega\text{发生时采取行动}a\text{的效用}}{u_{R}\left(a,\omega \right)} $$\pause
	令$\hat{a}$表示均衡时信号接受者$R$的行动,
	$$\hat{a}\left(q \right) \ \in \underset{a\in A}{\arg\max}\ \  E_{q}\left[ u_{S} \left( a,\omega\right) \right]$$
\end{frame}

\begin{frame}{信号发送者的期望收益}
	给定后验信念$q$和信号接受者行动$\hat{a}\left(q\right)$,令$\hat{v}\left(q\right)$为信号发送者$R$的期望收益:
	$$\hat{v}\left(q\right)=E_{q}\left[ u_{S} \left( \hat{a}\left(q\right),\omega\right) \right]$$
\end{frame}

\subsection{转译例子}

\begin{frame}{基准假设}
	状态空间$\Omega = \left\lbrace G,I\right\rbrace,\ \omega\in\Omega $\\
	信息发送者:检察官(prosecutor)\\
	信息接受者:法官(judge)\\
	行动集合:$A=\left\lbrace \text{判无罪(acquit)},\text{判有罪(convict)}\right\rbrace \pause$\\
	先验概率:$\Pr\left( G\right) =0.3$,$ \Pr\left( I\right) =0.7 $\\
	信号接受者$R$的效用函数:
	$$u_{R}\left( acquit,I\right) =u_{R}\left( convict,G\right)=1 \text{;} u_{R}\left( acquit,G\right) =u_{R}\left( convict,I\right)=0$$
	信号发送者$S$的效用函数:
	$$u_{S}\left( acquit,I\right) =u_{S}\left( acquit,G\right)=0 \text{;} u_{S}\left( convict,G\right) =u_{S}\left( convict,I\right)=0$$
\end{frame}

\begin{frame}{后验信念}
	$$a\left(q\right)=\lbrace \begin{array}{ll}
		\{acquit\} & if\ q\left(G\right)<\frac{1}{2} \\
		\{convict\} & if\ q\left(G\right)>\frac{1}{2}
	\end{array}$$
\end{frame}

\section{例子:王明等(2021,世界经济)}

\begin{frame}{问题导入}
	治理失灵现象普遍存在,但治理失灵被内生于治理体系中。\\~\\
	\begin{itemize}
		\item 基层政府的选择性执行
		\item 政策治理的变通行为
		\item 消极执行与运动式执行 \\~\\
	\end{itemize}
	
	“非正式行为”为什么在上级政府具有纠偏能力时也会被制度化?\\
	
\end{frame}

\subsection{基准模型设定}

\begin{frame}
	假定上级政府$U$和下级政府$D$政策集为$X\in\{a,b\}$\\
	上级政府负责制定政策,下级政府负责执行政策。\\
\end{frame}
\subsection{严格执行}

\subsection{变通执行}

\section{贝叶斯劝说:正规表述}

\end{document}