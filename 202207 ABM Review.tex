\documentclass[aspectratio=169]{beamer}
\usepackage[english]{babel}
\usepackage{ctex}

% 目录标数字
\setbeamertemplate{section in toc}[sections numbered] 
% 无序列表用实心点
\setbeamertemplate{itemize item}{$\bullet$}
% 设置每页标题格式
\setbeamertemplate{frametitle}
{\vspace{-0.5cm}
	\insertframetitle
	\vspace{-0.5cm}}
% 去掉下面没用的导航条
\setbeamertemplate{navigation symbols}{}
% 设置页脚格式
\makeatother
\setbeamertemplate{footline}
{
	\leavevmode%
	\hbox{%
		\begin{beamercolorbox}[wd=.4\paperwidth,ht=2.25ex,dp=1ex,center]{author in head/foot}%
			\usebeamerfont{author in head/foot}\insertshortauthor
		\end{beamercolorbox}
		
		\begin{beamercolorbox}[wd=.6\paperwidth,ht=2.25ex,dp=1ex,center]{title in head/foot}%
			\usebeamerfont{title in head/foot}\insertshorttitle\hspace*{13em}
			\insertframenumber{} / \inserttotalframenumber\hspace*{0ex}
	\end{beamercolorbox}}
	
	\vskip0pt%
}
\makeatletter


% 定义颜色
%\definecolor{alizarin}{rgb}{0.82, 0.1, 0.26} % 红色
%\definecolor{DarkFern}{HTML}{407428} % 绿色
%\colorlet{main}{DarkFern!100!white} % 第一种设置方法
%\colorlet{main}{red!70!black} % 第二种设置方法
\definecolor{bistre}{rgb}{0.24, 0.17, 0.12} % 黑色
\definecolor{mygrey}{rgb}{0.52, 0.52, 0.51} % 灰色
\colorlet{main}{green!50!black}
\colorlet{text}{bistre!100!white}

% 不同元素指定不同颜色,fg是本身颜色,bg是背景颜色,!num!改变数值提供渐变色
\setbeamercolor{title}{fg=main}
\setbeamercolor{frametitle}{fg=main}
\setbeamercolor{section in toc}{fg=text}
\setbeamercolor{normal text}{fg=text}
\setbeamercolor{block title}{fg=main,bg=mygrey!14!white}
\setbeamercolor{block body}{fg=black,bg=mygrey!10!white}
\setbeamercolor{qed symbol}{fg=main} % 证明结束后的框颜色
\setbeamercolor{math text}{fg=black}
% 设置页脚对应位置颜色
\setbeamercolor{author in head/foot}{fg=black, bg=mygrey!5!white}
\setbeamercolor{title in head/foot}{fg=black, bg=mygrey!5!white}
\setbeamercolor{structure}{fg=main, bg=mygrey!10!white} % 设置sidebar颜色

% 左右页间距的排版
\def\swidth{2.3cm}
\setbeamersize{sidebar width right=\swidth}
\setbeamersize{sidebar width left=\swidth}
\setbeamerfont{title in sidebar}{size=\scriptsize}
\setbeamerfont{section in sidebar}{size=\tiny}


%-------------------正文-------------------------%

\author{赵万春,天津财经大学}
\title{Agent-Based Modeling in Economics and Finance: 
	Past, Present, and Future}
\date{July, 2022}

\begin{document}
	
	\frame[plain]{\titlepage}
	
	\begin{frame}
		\frametitle{Outline}
		\tableofcontents
	\end{frame}
	
	\section{Introduction}
	
	\frame{\frametitle{Outline}\tableofcontents[currentsection]}
	
	\begin{frame}
		\frametitle{Introduction:What is ABM?}
		
		
		
		\vspace{0.4cm}
		
		
		
		
		unordered list below
		
		\begin{itemize}
			\item The first item
			\item The second item
			\item The third item
			\item The fourth item
		\end{itemize}
		
	\end{frame}
	
	\section{Display Theorem}
	
	\frame{\frametitle{Outline}\tableofcontents[currentsection]}
	
	\subsection{first subsection}
	
	\subsection{second subsection}
	
	
	
	\begin{frame}
		\frametitle{Display Theorem}
		\begin{theorem}
			$1 + 2 = 3$
		\end{theorem}
		\begin{proof}
			$$1 + 1 = 2$$
			$$1 + 1 + 1 = 3$$
		\end{proof}
	\end{frame}
	
	\section{Sample frame title}
	
	\frame{\frametitle{Outline}\tableofcontents[currentsection]}
	
	\begin{frame}
		\frametitle{Sample frame title}
		This is a text in second frame.
		For the sake of showing an example.
		
		\begin{itemize}
			\item Text visible on slide 1
			\item Text visible on slide 2
			\item Text visible on slide 3
		\end{itemize}
		
		\vspace{0.3cm}
		
		another example
		
		\begin{itemize}\itemsep0em
			\item Text visible on slide 1
			\item Text visible on slide 2
			\item Text visible on slide 3
		\end{itemize}
	\end{frame}
	
	\begin{frame}
		\begin{proof}
			$$
			\frac{1}{\displaystyle 1+
				\frac{1}{\displaystyle 2+
					\frac{1}{\displaystyle 3+x}}} +
			\frac{1}{1+\frac{1}{2+\frac{1}{3+x}}}
			$$
			$$\int_0^\infty e^{-x^2} dx=\frac{\sqrt{\pi}}{2}$$
			\begin{equation} x=y+3 \label{eq:xdef}
			\end{equation}
			In equation (\ref{eq:xdef}) we saw $\dots$
		\end{proof}
	\end{frame}
	
	
\end{document}